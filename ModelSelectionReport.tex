% Options for packages loaded elsewhere
\PassOptionsToPackage{unicode}{hyperref}
\PassOptionsToPackage{hyphens}{url}
%
\documentclass[
]{article}
\usepackage{amsmath,amssymb}
\usepackage{iftex}
\ifPDFTeX
  \usepackage[T1]{fontenc}
  \usepackage[utf8]{inputenc}
  \usepackage{textcomp} % provide euro and other symbols
\else % if luatex or xetex
  \usepackage{unicode-math} % this also loads fontspec
  \defaultfontfeatures{Scale=MatchLowercase}
  \defaultfontfeatures[\rmfamily]{Ligatures=TeX,Scale=1}
\fi
\usepackage{lmodern}
\ifPDFTeX\else
  % xetex/luatex font selection
\fi
% Use upquote if available, for straight quotes in verbatim environments
\IfFileExists{upquote.sty}{\usepackage{upquote}}{}
\IfFileExists{microtype.sty}{% use microtype if available
  \usepackage[]{microtype}
  \UseMicrotypeSet[protrusion]{basicmath} % disable protrusion for tt fonts
}{}
\makeatletter
\@ifundefined{KOMAClassName}{% if non-KOMA class
  \IfFileExists{parskip.sty}{%
    \usepackage{parskip}
  }{% else
    \setlength{\parindent}{0pt}
    \setlength{\parskip}{6pt plus 2pt minus 1pt}}
}{% if KOMA class
  \KOMAoptions{parskip=half}}
\makeatother
\usepackage{xcolor}
\usepackage[margin=1in]{geometry}
\usepackage{color}
\usepackage{fancyvrb}
\newcommand{\VerbBar}{|}
\newcommand{\VERB}{\Verb[commandchars=\\\{\}]}
\DefineVerbatimEnvironment{Highlighting}{Verbatim}{commandchars=\\\{\}}
% Add ',fontsize=\small' for more characters per line
\usepackage{framed}
\definecolor{shadecolor}{RGB}{248,248,248}
\newenvironment{Shaded}{\begin{snugshade}}{\end{snugshade}}
\newcommand{\AlertTok}[1]{\textcolor[rgb]{0.94,0.16,0.16}{#1}}
\newcommand{\AnnotationTok}[1]{\textcolor[rgb]{0.56,0.35,0.01}{\textbf{\textit{#1}}}}
\newcommand{\AttributeTok}[1]{\textcolor[rgb]{0.13,0.29,0.53}{#1}}
\newcommand{\BaseNTok}[1]{\textcolor[rgb]{0.00,0.00,0.81}{#1}}
\newcommand{\BuiltInTok}[1]{#1}
\newcommand{\CharTok}[1]{\textcolor[rgb]{0.31,0.60,0.02}{#1}}
\newcommand{\CommentTok}[1]{\textcolor[rgb]{0.56,0.35,0.01}{\textit{#1}}}
\newcommand{\CommentVarTok}[1]{\textcolor[rgb]{0.56,0.35,0.01}{\textbf{\textit{#1}}}}
\newcommand{\ConstantTok}[1]{\textcolor[rgb]{0.56,0.35,0.01}{#1}}
\newcommand{\ControlFlowTok}[1]{\textcolor[rgb]{0.13,0.29,0.53}{\textbf{#1}}}
\newcommand{\DataTypeTok}[1]{\textcolor[rgb]{0.13,0.29,0.53}{#1}}
\newcommand{\DecValTok}[1]{\textcolor[rgb]{0.00,0.00,0.81}{#1}}
\newcommand{\DocumentationTok}[1]{\textcolor[rgb]{0.56,0.35,0.01}{\textbf{\textit{#1}}}}
\newcommand{\ErrorTok}[1]{\textcolor[rgb]{0.64,0.00,0.00}{\textbf{#1}}}
\newcommand{\ExtensionTok}[1]{#1}
\newcommand{\FloatTok}[1]{\textcolor[rgb]{0.00,0.00,0.81}{#1}}
\newcommand{\FunctionTok}[1]{\textcolor[rgb]{0.13,0.29,0.53}{\textbf{#1}}}
\newcommand{\ImportTok}[1]{#1}
\newcommand{\InformationTok}[1]{\textcolor[rgb]{0.56,0.35,0.01}{\textbf{\textit{#1}}}}
\newcommand{\KeywordTok}[1]{\textcolor[rgb]{0.13,0.29,0.53}{\textbf{#1}}}
\newcommand{\NormalTok}[1]{#1}
\newcommand{\OperatorTok}[1]{\textcolor[rgb]{0.81,0.36,0.00}{\textbf{#1}}}
\newcommand{\OtherTok}[1]{\textcolor[rgb]{0.56,0.35,0.01}{#1}}
\newcommand{\PreprocessorTok}[1]{\textcolor[rgb]{0.56,0.35,0.01}{\textit{#1}}}
\newcommand{\RegionMarkerTok}[1]{#1}
\newcommand{\SpecialCharTok}[1]{\textcolor[rgb]{0.81,0.36,0.00}{\textbf{#1}}}
\newcommand{\SpecialStringTok}[1]{\textcolor[rgb]{0.31,0.60,0.02}{#1}}
\newcommand{\StringTok}[1]{\textcolor[rgb]{0.31,0.60,0.02}{#1}}
\newcommand{\VariableTok}[1]{\textcolor[rgb]{0.00,0.00,0.00}{#1}}
\newcommand{\VerbatimStringTok}[1]{\textcolor[rgb]{0.31,0.60,0.02}{#1}}
\newcommand{\WarningTok}[1]{\textcolor[rgb]{0.56,0.35,0.01}{\textbf{\textit{#1}}}}
\usepackage{graphicx}
\makeatletter
\def\maxwidth{\ifdim\Gin@nat@width>\linewidth\linewidth\else\Gin@nat@width\fi}
\def\maxheight{\ifdim\Gin@nat@height>\textheight\textheight\else\Gin@nat@height\fi}
\makeatother
% Scale images if necessary, so that they will not overflow the page
% margins by default, and it is still possible to overwrite the defaults
% using explicit options in \includegraphics[width, height, ...]{}
\setkeys{Gin}{width=\maxwidth,height=\maxheight,keepaspectratio}
% Set default figure placement to htbp
\makeatletter
\def\fps@figure{htbp}
\makeatother
\setlength{\emergencystretch}{3em} % prevent overfull lines
\providecommand{\tightlist}{%
  \setlength{\itemsep}{0pt}\setlength{\parskip}{0pt}}
\setcounter{secnumdepth}{-\maxdimen} % remove section numbering
\ifLuaTeX
  \usepackage{selnolig}  % disable illegal ligatures
\fi
\usepackage{bookmark}
\IfFileExists{xurl.sty}{\usepackage{xurl}}{} % add URL line breaks if available
\urlstyle{same}
\hypersetup{
  pdftitle={Model Selection Report},
  pdfauthor={Loïc Pages},
  hidelinks,
  pdfcreator={LaTeX via pandoc}}

\title{Model Selection Report}
\author{Loïc Pages}
\date{2025-03-27}

\begin{document}
\maketitle

\section{Introduction}\label{introduction}

\subsection{IPM}\label{ipm}

\[N(y,t+1)=\int N(x,t)(F_a(x,y,t)+P_a(x,y,t))dx\] avec \(x\) la taille à
\(t\), \(y\) la taille à \(t+1\), \(a\) l'age et \(N\) la taille de la
population

La probabilité qu'un individu d'age \(a\) et de taille \(x\) au temps
\(t\) devienne un individu d'age \(a+1\) et de taille \(y\) à \(t+1\)
est : \[P_a(x,y,t)=s_a(x)(1-f_a(x))G_a(x,y)\] et la densité d'individus
de taille \(y\) à \(t\) et d'age \(1\) (plantules) issus d'un individu
de taille \(x\) et d'age \(a\) à \(t\) est :
\[F_a(x,y,t)=f_a(x)C_a(x)w(y)Estb\]

\subsection{Sélection de modèles}\label{suxe9lection-de-moduxe8les}

Méthode utilisée: Créer l'ensemble des combinaisons possibles d'effets
fixes et aléatoires. Puis utiliser le package spaMM pour fit les
modèles. On calcule ensuite leur AIC (ou leur BIC) qu'on compare entre
eux. On obtient un classement des (5) meilleurs modèles pour chaque
trait d'histoire de vie, pour l'AIC et pour le BIC.

Paramètres à modéliser :

Survival Probability

Flowering Probability

Growth

Fecundity (Number of capitula)

Seedling Size Distribution

\subsection{Initialisation}\label{initialisation}

\begin{Shaded}
\begin{Highlighting}[]
\FunctionTok{rm}\NormalTok{(}\AttributeTok{list=}\FunctionTok{ls}\NormalTok{())}
\FunctionTok{library}\NormalTok{(knitr)}
\FunctionTok{library}\NormalTok{(spaMM)}
\FunctionTok{library}\NormalTok{(tidyverse)}
\FunctionTok{library}\NormalTok{(splines)}
\FunctionTok{library}\NormalTok{(foreach)}
\FunctionTok{library}\NormalTok{(doParallel)}

\FunctionTok{setwd}\NormalTok{(}\StringTok{"/media/loic/Commun/0Travail/Stage 2025 ISEM/Models"}\NormalTok{)}

\NormalTok{centauree\_data }\OtherTok{\textless{}{-}} \FunctionTok{read.csv}\NormalTok{(}\StringTok{"donnesIPM\_short.csv"}\NormalTok{)}
\NormalTok{centauree\_data\_complet }\OtherTok{\textless{}{-}} \FunctionTok{read.csv}\NormalTok{(}\StringTok{"donnesIPM.csv"}\NormalTok{)}

\CommentTok{\#Supprimer plantes dont l\textquotesingle{}age est inconnu}
\NormalTok{centauree\_data }\OtherTok{\textless{}{-}}\NormalTok{ centauree\_data[}\SpecialCharTok{!}\FunctionTok{is.na}\NormalTok{(centauree\_data}\SpecialCharTok{$}\NormalTok{age0), ]}

\CommentTok{\#Ajouter valeur de l\textquotesingle{}age à t+1}
\NormalTok{centauree\_data}\SpecialCharTok{$}\NormalTok{age1 }\OtherTok{\textless{}{-}} \FunctionTok{ifelse}\NormalTok{(centauree\_data}\SpecialCharTok{$}\NormalTok{Stage1}\SpecialCharTok{==}\StringTok{"V"}\NormalTok{,centauree\_data}\SpecialCharTok{$}\NormalTok{age0}\SpecialCharTok{+}\DecValTok{1}\NormalTok{,}\ConstantTok{NA}\NormalTok{)}

\CommentTok{\#Forcer l\textquotesingle{}age maximal à 8}
\NormalTok{centauree\_data}\SpecialCharTok{$}\NormalTok{age0[centauree\_data}\SpecialCharTok{$}\NormalTok{age0 }\SpecialCharTok{\textgreater{}} \DecValTok{8}\NormalTok{] }\OtherTok{\textless{}{-}} \DecValTok{8}
\NormalTok{centauree\_data}\SpecialCharTok{$}\NormalTok{age1[centauree\_data}\SpecialCharTok{$}\NormalTok{age1 }\SpecialCharTok{\textgreater{}} \DecValTok{8}\NormalTok{] }\OtherTok{\textless{}{-}} \DecValTok{8}

\FunctionTok{spaMM.options}\NormalTok{(}\AttributeTok{separation\_max=}\DecValTok{70}\NormalTok{)}
\end{Highlighting}
\end{Shaded}

\section{Mise en place}\label{mise-en-place}

\subsection{Survival Probability}\label{survival-probability}

Pour le probabilité de survie, on a divisé le jeu de données en deux:

1- les plantules (individus d'age 1)

2- les rosettes (individus d'age 2 ou plus)

(La somme des AIC des meilleurs modèles de survie des plantules et des
rosettes est inférieure à l'AIC des modèles de survie comprenant tous
les ages)

On utilise un modèle binomial pour fit les données de survie (0 ou 1).

Affichage des données brutes de la survie pour tous les ages.

\includegraphics[width=0.5\linewidth]{ModelSelectionReport_files/figure-latex/unnamed-chunk-4-1}
\includegraphics[width=0.5\linewidth]{ModelSelectionReport_files/figure-latex/unnamed-chunk-4-2}

\subsubsection{Age 1 - Plantules}\label{age-1---plantules}

Pour la survie des plantules, les effets explorés sont :

fixe : taille (polynomes et splines)

aléatoire : population, année (effet sur l'intercept ou la taille)

\includegraphics[width=0.5\linewidth]{ModelSelectionReport_files/figure-latex/unnamed-chunk-5-1}

\subsubsection{Age +2 - Rosettes}\label{age-2---rosettes}

Pour la survie des rosettes, les effets explorés sont :

fixe : taille (polynomes et splines) et age (polynome et splines)

aléatoire : population, année (effet sur l'intercept ou la taille),
hétérogénéité individuelle (intercept)

\includegraphics[width=0.5\linewidth]{ModelSelectionReport_files/figure-latex/unnamed-chunk-6-1}
\includegraphics[width=0.5\linewidth]{ModelSelectionReport_files/figure-latex/unnamed-chunk-6-2}

\subsection{Flowering Probability}\label{flowering-probability}

On utilise un modèle binomial pour fit les données de floraison (0 ou
1).

Pour les modèles de floraison, les possibilités de combinaisons d'effets
fixes et aléatoires ont été réduit pour simplifier les calculs de
sélection.

Les effets d'hétérogénéité entre individus ont été retirés par soucis de
temps de calcul et de convergence des modèles.

Les effets fixes de l'age ont été simplifiés en retirant les splines.

D'autres combinaisons (notamment contenant les splines de la taille) ont
été retirés par un algorithme détectant si le modèle subit des effets de
séparation qui empêchent la convergence du modèle.

\includegraphics[width=0.5\linewidth]{ModelSelectionReport_files/figure-latex/unnamed-chunk-7-1}
\includegraphics[width=0.5\linewidth]{ModelSelectionReport_files/figure-latex/unnamed-chunk-7-2}

\subsection{Seedl Size}\label{seedl-size}

Pour fit les données de taille de plantules, on utilise un modèle qui
suit une distribution log-Gamma.

On y fait varier les effets aléatoires années, population et
l'intéraction année:population.

\includegraphics[width=0.5\linewidth]{ModelSelectionReport_files/figure-latex/unnamed-chunk-8-1}
\includegraphics[width=0.5\linewidth]{ModelSelectionReport_files/figure-latex/unnamed-chunk-8-2}

\begin{verbatim}
## `stat_bin()` using `bins = 30`. Pick better value with `binwidth`.
\end{verbatim}

\begin{verbatim}
## Warning: Removed 3 rows containing non-finite outside the scale range
## (`stat_bin()`).
\end{verbatim}

\begin{verbatim}
## Warning: Removed 2 rows containing missing values or values outside the scale range
## (`geom_bar()`).
\end{verbatim}

\includegraphics[width=0.5\linewidth]{ModelSelectionReport_files/figure-latex/unnamed-chunk-8-3}

\subsection{Growth}\label{growth}

Pour la croissance des plantes, on a choisi de fit le log de la taille à
l'année \(t+1\) en fonction des effets fixes log de la taille à l'année
\(t\) et age, et des effets aléatoires population, année et
hétérogénéité individuelle.

La taille à l'année \(t+1\) suit une distribution logNormale.

\includegraphics[width=0.5\linewidth]{ModelSelectionReport_files/figure-latex/unnamed-chunk-9-1}
\includegraphics[width=0.5\linewidth]{ModelSelectionReport_files/figure-latex/unnamed-chunk-9-2}
\includegraphics[width=0.5\linewidth]{ModelSelectionReport_files/figure-latex/unnamed-chunk-9-3}

\subsection{Fecondity}\label{fecondity}

\includegraphics[width=0.5\linewidth]{ModelSelectionReport_files/figure-latex/unnamed-chunk-10-1}
\includegraphics[width=0.5\linewidth]{ModelSelectionReport_files/figure-latex/unnamed-chunk-10-2}

\section{Modèles par AIC}\label{moduxe8les-par-aic}

\begin{Shaded}
\begin{Highlighting}[]
\NormalTok{Survglm11 }\OtherTok{\textless{}{-}} \FunctionTok{fitme}\NormalTok{(SurvieMars }\SpecialCharTok{\textasciitilde{}} \DecValTok{1} \SpecialCharTok{+} \FunctionTok{poly}\NormalTok{(Size0Mars,}\DecValTok{3}\NormalTok{) }\SpecialCharTok{+} 
\NormalTok{                                  (Size0Mars}\SpecialCharTok{|}\NormalTok{year) }\SpecialCharTok{+}\NormalTok{ (}\DecValTok{1}\SpecialCharTok{|}\NormalTok{Pop),}
                  \AttributeTok{family=}\NormalTok{binomial,}
                  \AttributeTok{data=}\NormalTok{survdata1,}
                  \AttributeTok{method=}\StringTok{"PQL/L"}\NormalTok{)}
\FunctionTok{extractAIC}\NormalTok{(Survglm11)}
\end{Highlighting}
\end{Shaded}

\begin{verbatim}
##      edf      AIC 
##    4.000 4373.984
\end{verbatim}

\begin{Shaded}
\begin{Highlighting}[]
\NormalTok{Survglm12 }\OtherTok{\textless{}{-}} \FunctionTok{fitme}\NormalTok{(SurvieMars }\SpecialCharTok{\textasciitilde{}} \DecValTok{1} \SpecialCharTok{+} \FunctionTok{bs}\NormalTok{(Size0Mars,}\AttributeTok{df=}\DecValTok{4}\NormalTok{,}\AttributeTok{degree=}\DecValTok{2}\NormalTok{) }\SpecialCharTok{+}\NormalTok{(}\FunctionTok{bs}\NormalTok{(age0,}\AttributeTok{degree=}\DecValTok{3}\NormalTok{,}\AttributeTok{knots=}\FloatTok{6.5}\NormalTok{)) }\SpecialCharTok{+} 
\NormalTok{                                  (age0}\SpecialCharTok{|}\NormalTok{year) }\SpecialCharTok{+}\NormalTok{ (}\DecValTok{1}\SpecialCharTok{|}\NormalTok{Pop),}
                  \AttributeTok{family=}\NormalTok{binomial,}
                  \AttributeTok{data=}\NormalTok{survdata2,}
                  \AttributeTok{method=}\StringTok{"PQL/L"}\NormalTok{)}
\FunctionTok{extractAIC}\NormalTok{(Survglm12)}
\end{Highlighting}
\end{Shaded}

\begin{verbatim}
##      edf      AIC 
##    9.000 2309.076
\end{verbatim}

\begin{Shaded}
\begin{Highlighting}[]
\NormalTok{Flowglm1 }\OtherTok{\textless{}{-}} \FunctionTok{fitme}\NormalTok{(Flowering0 }\SpecialCharTok{\textasciitilde{}}  \DecValTok{1} \SpecialCharTok{+} \FunctionTok{poly}\NormalTok{(Size0Mars,}\DecValTok{3}\NormalTok{) }\SpecialCharTok{+} \FunctionTok{poly}\NormalTok{(age0,}\DecValTok{2}\NormalTok{) }\SpecialCharTok{+} 
\NormalTok{                    (age0}\SpecialCharTok{|}\NormalTok{Pop),}
                 \AttributeTok{family=}\NormalTok{binomial,}
                 \AttributeTok{data=}\NormalTok{centauree\_data, }\AttributeTok{method=}\StringTok{"PQL/L"}\NormalTok{)}
\FunctionTok{extractAIC}\NormalTok{(Flowglm1)}
\end{Highlighting}
\end{Shaded}

\begin{verbatim}
##     edf     AIC 
##   6.000 843.909
\end{verbatim}

\begin{Shaded}
\begin{Highlighting}[]
\NormalTok{Pltglm1 }\OtherTok{\textless{}{-}} \FunctionTok{fitme}\NormalTok{(Size0Mars }\SpecialCharTok{\textasciitilde{}} \DecValTok{1} \SpecialCharTok{+}\NormalTok{ (}\DecValTok{1}\SpecialCharTok{|}\NormalTok{year) }\SpecialCharTok{+}\NormalTok{ (}\DecValTok{1}\SpecialCharTok{|}\NormalTok{Pop) }\SpecialCharTok{+}\NormalTok{ (}\DecValTok{1}\SpecialCharTok{|}\NormalTok{Pop}\SpecialCharTok{:}\NormalTok{year), }
                 \AttributeTok{data=}\NormalTok{plantule\_data,}
                 \AttributeTok{family =} \FunctionTok{Gamma}\NormalTok{(log))}
\FunctionTok{extractAIC}\NormalTok{(Pltglm1)}
\end{Highlighting}
\end{Shaded}

\begin{verbatim}
##      edf      AIC 
##    1.000 8442.641
\end{verbatim}

\begin{Shaded}
\begin{Highlighting}[]
\NormalTok{Growthglm1 }\OtherTok{\textless{}{-}} \FunctionTok{fitme}\NormalTok{(}\FunctionTok{log}\NormalTok{(Size1Mars) }\SpecialCharTok{\textasciitilde{}} \DecValTok{1} \SpecialCharTok{+} 
                      \FunctionTok{poly}\NormalTok{(Size0Mars,}\DecValTok{4}\NormalTok{) }\SpecialCharTok{+} \FunctionTok{poly}\NormalTok{(age0,}\DecValTok{3}\NormalTok{) }\SpecialCharTok{+} 
\NormalTok{                      (Size0Mars}\SpecialCharTok{|}\NormalTok{year) }\SpecialCharTok{+}\NormalTok{ (Size0Mars}\SpecialCharTok{|}\NormalTok{Pop),}
                    \AttributeTok{data=}\NormalTok{growthdata,}\AttributeTok{resid.model =} \SpecialCharTok{\textasciitilde{}}\FunctionTok{log}\NormalTok{(Size0Mars))}
\FunctionTok{extractAIC}\NormalTok{(Growthglm1)}
\end{Highlighting}
\end{Shaded}

\begin{verbatim}
##      edf      AIC 
##    8.000 3878.299
\end{verbatim}

\begin{Shaded}
\begin{Highlighting}[]
\NormalTok{Growthglm2 }\OtherTok{\textless{}{-}} \FunctionTok{fitme}\NormalTok{(Size1Mars }\SpecialCharTok{\textasciitilde{}} \DecValTok{1} \SpecialCharTok{+}
                      \FunctionTok{bs}\NormalTok{(Size0Mars,}\AttributeTok{degree=}\DecValTok{3}\NormalTok{,}\AttributeTok{df=}\DecValTok{5}\NormalTok{) }\SpecialCharTok{+} \FunctionTok{bs}\NormalTok{(age0,}\AttributeTok{degree=}\DecValTok{2}\NormalTok{,}\AttributeTok{knots=}\FloatTok{6.5}\NormalTok{) }\SpecialCharTok{+}
\NormalTok{                      (Size0Mars}\SpecialCharTok{+}\NormalTok{age0}\SpecialCharTok{|}\NormalTok{year) }\SpecialCharTok{+}\NormalTok{ (Size0Mars}\SpecialCharTok{|}\NormalTok{Pop),}
                    \AttributeTok{resid.model =} \SpecialCharTok{\textasciitilde{}} \FunctionTok{log}\NormalTok{(Size0Mars),}
                    \AttributeTok{data=}\NormalTok{growthdata)}

\NormalTok{cb }\OtherTok{\textless{}{-}}\NormalTok{ MASS}\SpecialCharTok{::}\FunctionTok{boxcox}\NormalTok{(Size1Mars }\SpecialCharTok{\textasciitilde{}} \DecValTok{1}\NormalTok{, }\AttributeTok{data=}\NormalTok{growthdata)}
\end{Highlighting}
\end{Shaded}

\includegraphics{ModelSelectionReport_files/figure-latex/unnamed-chunk-11-1.pdf}

\begin{Shaded}
\begin{Highlighting}[]
\NormalTok{cb}\SpecialCharTok{$}\NormalTok{x[cb}\SpecialCharTok{$}\NormalTok{y}\SpecialCharTok{==}\FunctionTok{max}\NormalTok{(cb}\SpecialCharTok{$}\NormalTok{y)]}
\end{Highlighting}
\end{Shaded}

\begin{verbatim}
## [1] 0.2626263
\end{verbatim}

\begin{Shaded}
\begin{Highlighting}[]
\NormalTok{Cptlglm1 }\OtherTok{\textless{}{-}} \FunctionTok{fitme}\NormalTok{(Cptl0 }\SpecialCharTok{\textasciitilde{}} \DecValTok{1} \SpecialCharTok{+}\NormalTok{ Size0Mars, }
                  \AttributeTok{data=}\NormalTok{cptldata)}
\FunctionTok{extractAIC}\NormalTok{(Cptlglm1)}
\end{Highlighting}
\end{Shaded}

\begin{verbatim}
##      edf      AIC 
##   2.0000 825.2362
\end{verbatim}

\begin{Shaded}
\begin{Highlighting}[]
\FunctionTok{library}\NormalTok{(SplinesUtils)}

\CommentTok{\#Pour un exemple :}
\CommentTok{\# RegModel \textless{}{-} fitme(SurvieMars \textasciitilde{} 1 + bs(Size0Mars, df = 4, degree = 2) +(bs(age0, degree = 3, knots = 6.5)) + (age0|year) + (1|Pop),}
\CommentTok{\#                   family=binomial,}
\CommentTok{\#                   data=survdata2,}
\CommentTok{\#                   method="PQL/L")}
\CommentTok{\# SplineTerm \textless{}{-} "bs(Size0Mars, df = 4, degree = 2)"}

\CommentTok{\# La fonction d\textquotesingle{}origine est RegSplineAsPiecePoly}

\NormalTok{SplinesAsPolySpaMM }\OtherTok{\textless{}{-}} \ControlFlowTok{function}\NormalTok{ (RegModel, SplineTerm, }\AttributeTok{shift =} \ConstantTok{TRUE}\NormalTok{) \{}
  
  \CommentTok{\#Extrait les données du spline (position dans les effets fixes, degrés des polynomes, knots)}
\NormalTok{  RegSpline }\OtherTok{\textless{}{-}}\NormalTok{ SplinesUtils}\SpecialCharTok{:::}\FunctionTok{ExtractSplineTerm}\NormalTok{(}\FunctionTok{terms}\NormalTok{(RegModel), SplineTerm)}
\NormalTok{  pos }\OtherTok{\textless{}{-}}\NormalTok{ RegSpline}\SpecialCharTok{$}\NormalTok{pos}
  
  \CommentTok{\#Permet de sortir les coefficients de chaque bout de splines}

\NormalTok{  ind0 }\OtherTok{\textless{}{-}} \FunctionTok{attr}\NormalTok{(RegModel}\SpecialCharTok{$}\NormalTok{X.pv, }\StringTok{"namesOri"}\NormalTok{) }\CommentTok{\#Noms des coefficients}
\NormalTok{  ind }\OtherTok{\textless{}{-}} \FunctionTok{integer}\NormalTok{(}\DecValTok{0}\NormalTok{)}
\NormalTok{  pattern }\OtherTok{\textless{}{-}} \FunctionTok{paste0}\NormalTok{(}\StringTok{"\^{}"}\NormalTok{, }\FunctionTok{gsub}\NormalTok{(}\StringTok{"}\SpecialCharTok{\textbackslash{}\textbackslash{}}\StringTok{("}\NormalTok{, }\StringTok{"}\SpecialCharTok{\textbackslash{}\textbackslash{}\textbackslash{}\textbackslash{}}\StringTok{("}\NormalTok{, }\FunctionTok{gsub}\NormalTok{(}\StringTok{"}\SpecialCharTok{\textbackslash{}\textbackslash{}}\StringTok{)"}\NormalTok{, }\StringTok{"}\SpecialCharTok{\textbackslash{}\textbackslash{}\textbackslash{}\textbackslash{}}\StringTok{)"}\NormalTok{, SplineTerm)), }\StringTok{"}\SpecialCharTok{\textbackslash{}\textbackslash{}}\StringTok{d+$"}\NormalTok{) }
  \CommentTok{\#C\textquotesingle{}est pas très propre mais pas le choix car les noms des bouts de splines finissent avec un indice 1,2,...}
\NormalTok{  ind }\OtherTok{\textless{}{-}} \FunctionTok{grep}\NormalTok{(pattern, ind0) }\CommentTok{\#Détecte si le nom du coefficient correspond au spline voulu}
\NormalTok{  RegSplineCoef }\OtherTok{\textless{}{-}}\NormalTok{ RegModel}\SpecialCharTok{$}\NormalTok{fixef[ind] }\CommentTok{\#Sort l\textquotesingle{}estimation des coeffs}
  
\NormalTok{  RegSplineCoef }\OtherTok{\textless{}{-}} \FunctionTok{unname}\NormalTok{(RegSplineCoef)}
\NormalTok{  na }\OtherTok{\textless{}{-}} \FunctionTok{is.na}\NormalTok{(RegSplineCoef)}
  \ControlFlowTok{if}\NormalTok{ (}\FunctionTok{any}\NormalTok{(na)) \{}
    \FunctionTok{warning}\NormalTok{(}\StringTok{"NA coefficients found for SplineTerm; Replacing NA by 0"}\NormalTok{)}
\NormalTok{    RegSplineCoef[na] }\OtherTok{\textless{}{-}} \DecValTok{0}
\NormalTok{  \}}
  
  \CommentTok{\#Recalcule le polynome avec la méthode d\textquotesingle{}origine, je n\textquotesingle{}y ai pas touché}
\NormalTok{  PiecePolyCoef }\OtherTok{\textless{}{-}}\NormalTok{ SplinesUtils}\SpecialCharTok{:::}\FunctionTok{PiecePolyRepara}\NormalTok{(RegSpline, RegSplineCoef, shift)}
  \FunctionTok{structure}\NormalTok{(}\FunctionTok{list}\NormalTok{(}
    \AttributeTok{PiecePoly =} \FunctionTok{list}\NormalTok{(}\AttributeTok{coef =}\NormalTok{ PiecePolyCoef, }\AttributeTok{shift =}\NormalTok{ shift),}
    \AttributeTok{knots =}\NormalTok{ RegSpline}\SpecialCharTok{$}\NormalTok{knots),}
  \AttributeTok{class =} \FunctionTok{c}\NormalTok{(}\StringTok{"PiecePoly"}\NormalTok{, }\StringTok{"RegSpline"}\NormalTok{))}
\NormalTok{\}}
\end{Highlighting}
\end{Shaded}

\begin{Shaded}
\begin{Highlighting}[]
\FunctionTok{SplinesAsPolySpaMM}\NormalTok{(Survglm12,}\StringTok{"bs(Size0Mars, df = 4, degree = 2)"}\NormalTok{)}
\end{Highlighting}
\end{Shaded}

\begin{verbatim}
## 3 piecewise polynomials of degree 2 are constructed!
## Use 'summary' to export all of them.
## The first 3 are printed below.
## 2.56e-16 + 1.28 * (x - 0.5) - 0.142 * (x - 0.5) ^ 2
## 2.31 + 0.569 * (x - 3) - 0.0708 * (x - 3) ^ 2
## 3.38 + 0.144 * (x - 6) - 0.00504 * (x - 6) ^ 2
\end{verbatim}

\begin{Shaded}
\begin{Highlighting}[]
\FunctionTok{SplinesAsPolySpaMM}\NormalTok{(Growthglm2,}\StringTok{"bs(Size0Mars, degree = 3, df = 5)"}\NormalTok{)}
\end{Highlighting}
\end{Shaded}

\begin{verbatim}
## 3 piecewise polynomials of degree 3 are constructed!
## Use 'summary' to export all of them.
## The first 3 are printed below.
## 2.19e-15 + 0.257 * (x - 0.5) + 0.609 * (x - 0.5) ^ 2 - 0.135 * (x - 0.5) ^ 3
## 1.3 + 1.17 * (x - 2) + 0.000897 * (x - 2) ^ 2 - 0.00156 * (x - 2) ^ 3
## 4.78 + 1.14 * (x - 5) - 0.0131 * (x - 5) ^ 2 - 0.00182 * (x - 5) ^ 3
\end{verbatim}

\begin{Shaded}
\begin{Highlighting}[]
\FunctionTok{SplinesAsPolySpaMM}\NormalTok{(Growthglm2,}\StringTok{"bs(age0, degree = 2, knots = 6.5)"}\NormalTok{)}
\end{Highlighting}
\end{Shaded}

\begin{verbatim}
## 2 piecewise polynomials of degree 2 are constructed!
## Use 'summary' to export all of them.
## The first 2 are printed below.
## 0 - 0.768 * (x - 1) + 0.104 * (x - 1) ^ 2
## -1.07 + 0.379 * (x - 6.5) - 0.661 * (x - 6.5) ^ 2
\end{verbatim}

\begin{Shaded}
\begin{Highlighting}[]
\NormalTok{f }\OtherTok{\textless{}{-}} \ControlFlowTok{function}\NormalTok{(x)\{}
  \FunctionTok{return}\NormalTok{((}\FloatTok{2.56e{-}16} \SpecialCharTok{+} \FloatTok{1.28} \SpecialCharTok{*}\NormalTok{ (x }\SpecialCharTok{{-}} \FloatTok{0.5}\NormalTok{) }\SpecialCharTok{{-}} \FloatTok{0.142} \SpecialCharTok{*}\NormalTok{ (x }\SpecialCharTok{{-}} \FloatTok{0.5}\NormalTok{) }\SpecialCharTok{\^{}} \DecValTok{2}\NormalTok{)}\SpecialCharTok{*}\FunctionTok{I}\NormalTok{(x}\SpecialCharTok{\textless{}}\DecValTok{3}\NormalTok{)}\SpecialCharTok{+}\NormalTok{(}\FloatTok{2.31} \SpecialCharTok{+} \FloatTok{0.569} \SpecialCharTok{*}\NormalTok{ (x }\SpecialCharTok{{-}} \DecValTok{3}\NormalTok{) }\SpecialCharTok{{-}} \FloatTok{0.0708} \SpecialCharTok{*}\NormalTok{ (x }\SpecialCharTok{{-}} \DecValTok{3}\NormalTok{) }\SpecialCharTok{\^{}} \DecValTok{2}\NormalTok{)}\SpecialCharTok{*}\FunctionTok{I}\NormalTok{(x}\SpecialCharTok{\textgreater{}=}\DecValTok{3} \SpecialCharTok{\&}\NormalTok{ x}\SpecialCharTok{\textless{}}\DecValTok{6}\NormalTok{)}\SpecialCharTok{+}\NormalTok{(}\FloatTok{3.38} \SpecialCharTok{+} \FloatTok{0.144} \SpecialCharTok{*}\NormalTok{ (x }\SpecialCharTok{{-}} \DecValTok{6}\NormalTok{) }\SpecialCharTok{{-}} \FloatTok{0.00504} \SpecialCharTok{*}\NormalTok{ (x }\SpecialCharTok{{-}} \DecValTok{6}\NormalTok{) }\SpecialCharTok{\^{}} \DecValTok{2}\NormalTok{)}\SpecialCharTok{*}\FunctionTok{I}\NormalTok{(x}\SpecialCharTok{\textgreater{}=}\DecValTok{6}\NormalTok{))}
\NormalTok{\}}
\end{Highlighting}
\end{Shaded}

\subsection{Détails des modèles}\label{duxe9tails-des-moduxe8les}

\subsubsection{Survie des plantules}\label{survie-des-plantules}

\[logit(s_1(x))=\beta_0+b_{0,pop}+b_{0,year}+(\beta_1+b_{1,pop})x+\beta_2x^2+\beta_3x^3\]

\[logit(s_1(x))=-1.51+b_{0,pop}+b_{0,year}+(36.02+b_{1,pop})x+12.04x^2+29.46x^3\]

avec :

\(x\) : taille au temps t

\(\beta_i\), \(i\in\{0,1,2,3\}\): : coefficients du polynome de la
taille

\(b_{0,pop} \sim \mathcal{N}(0,0.89)\) : contribution de l'effet
aléatoire Population sur l'intercept

\(b_{j,year}\), \(j\in\{0,1\}\): contribution de l'effet aléatoire Année
sur l'intercept et la taille
\((b_{0,year},b_{1,year}) \sim \mathcal{N}(0,\Sigma)\) with \[\Sigma =
  \left[ {\begin{array}{cc}
    1.57 & -0.08 \\
    -0.08 & 0.066 \\
  \end{array} } \right]\]

Correlation term: \(\rho=-0.78\)

\subsubsection{Survie des rosettes}\label{survie-des-rosettes}

\[logit(s_a(x))=s_1(x)+s_2(a)\] \[f_1(x) = \left\{
    \begin{array}{ll}
        b_{0,pop}+b_{0,year}+1.28(x-0.5)-(0.142+b_{0,year})(x - 0.5) ^ 2 & \mbox{si } x \leq 2 \\
        b_{0,pop}+b_{0,year}+2.31+(0.569+b_{0,year})(x-3)-0.0708(x-3)^2 & \mbox{si } x \in [2;6] \\
        b_{0,pop}+b_{0,year}+3.38 + (0.144+b_{0,year})(x - 6)-0.00504(x-6)^2 & \mbox{si } x \geq 6
    \end{array}
\right.\]

\[f_2(a) = \left\{
    \begin{array}{ll}
        b_{0,pop}+0.612 (a-2) - 0.416 (a-2)^2 + 0.0617 (a-2)^3 & \mbox{si } a \leq 6.5 \\
        b_{0,pop}-0.0382 + 0.621 (a-6.5)+0.418(a-6.5)^2 -0.844(a-6.5)^3 & \mbox{si } x \geq 6.5 \\
    \end{array}
\right.\]

avec :

\(x\) : taille au temps \(t\)

\(a\) : age au temps \(t\)

\(b_{0,pop} \sim \mathcal{N}(0,0.073)\) : contribution de l'effet
aléatoire Population sur l'intercept

\(b_{j,year}\), \(j\in\{0,1\}\): contribution de l'effet aléatoire Année
sur l'intercept et l'age
\((b_{0,year},b_{1,year}) \sim \mathcal{N}(0,\Sigma)\) with \[\Sigma =
  \left[ {\begin{array}{cc}
    1.39 & -0.01 \\
    -0.01 & 0.0097 \\
  \end{array} } \right]\]

Correlation term: \(\rho=-0.91\)

\subsubsection{Floraison}\label{floraison}

\[logit(f_a(x))=\beta_0+b_{0,pop}+\beta_{1,1}x+\beta_{1,2}x^2+\beta_{1,3}x^3+(\beta_{2,1}+b_{1,pop})a+\beta_{2,2}a^2\]

\[logit(f_a(x))=11.49+b_{0,pop}+258.56x-86.88x^2+42.73x^3+(148.26+b_{1,pop})a-57.46a^2\]
avec :

\(a\) : age au temps \(t\)

\(x\) : taille au temps \(t\)

\(\beta_0\) : intercept des effets fixes

\(\beta_{1,i}\), \(i\in\{1,2,3\}\): coefficients du polynome de la
taille

\(\beta_{2,i}\), \(i\in\{1,2\}\): coefficients du polynome de l'age

\(b_{0,pop}\) : contribution de l'effet aléatoire Population sur
l'intercept

\(b_{1,pop}\) : contribution de l'effet aléatoire Population sur l'age

\((b_{0,pop},b_{1,pop}) \sim \mathcal{N}(0,\Sigma)\) with \[\Sigma =
  \left[ {\begin{array}{cc}
    2.17 & -0.19 \\
    -0.19 & 0.09 \\
  \end{array} } \right]\]

\begin{verbatim}
Correlation term:
\end{verbatim}

\(\rho=-0.98\)

\subsubsection{Taille des plantules}\label{taille-des-plantules}

\[y\sim\text{Gamma}(\mu,\sigma)\] avec
\(\mu=\exp(\beta_0+\gamma_0+\varepsilon_0+\delta_0)\) et
\(\sigma=phi\times\mu^2=0.32\)

\(y\) : taille de la plantule

\(\beta_0=0.21\),

\(\gamma_0\sim\mathcal{N}(0,0.022)\) : effet année

\(\varepsilon_0\sim\mathcal{N}(0,0.026)\) : effet population

\(\delta_0\sim\mathcal{N}(0,0.079)\) : effet d'intéraction
population:année

Residual variation : \(\phi=0.26\)

\subsubsection{Croissance}\label{croissance}

\paragraph{log-log}\label{log-log}

\[y\sim \mathcal{N}(\mu(x,a),\sigma(x))\]

avec
\(\mu(x,a)=\beta_0+b_{0,1}+b_{0,2}+(\beta_{1,1}+b_{1,1}+b_{1,2})x+\beta_{2,1}x^2+\beta_{3,1}x^3+\beta_{4,1}x^4+\beta_{2,1}a+\beta_{2,2}a^2+\beta_{3,2}a^3\)
\(\sigma(x)=-0.9 -0.45 \log(x)\)

\paragraph{1-1}\label{section}

\[\mu(x,a)=\mu_1(x)+\mu_2(a)\] \[\mu_1(x) = \left\{
    \begin{array}{ll}
        b_{0,1,year}+b_{0,pop} + (0.257+b_{1,1,year}+b_{1,pop})(x-0.5) + 0.609(x-0.5)^2 - 0.135(x-0.5)^3 & \mbox{si } x \leq 2 \\
         b_{0,1,year}+b_{0,pop}+1.3 + (1.17+b_{1,1,year}+b_{1,pop})(x-2) + 0.000897(x-2)^2 - 0.00156(x-2)^3 & \mbox{si } x \in [2;5] \\
         b_{0,1,year}+b_{0,pop}+4.78 + (1.14+b_{1,1,year}+b_{1,pop})(x-5) - 0.0131(x-5)^2 - 0.00182(x-5)^3 & \mbox{si } x \geq 5
    \end{array}
\right.\]

\[\mu_2(a) = \left\{
    \begin{array}{ll}
        - 0.768 (x - 1) + 0.104(x - 1) ^ 2 & \mbox{si } a \leq 6.5 \\
        -1.07 + 0.379(x - 6.5) - 0.661(x - 6.5) ^ 2 & \mbox{si } x \geq 6.5 \\
    \end{array}
\right.\]

\(K_{year} \sim \mathcal{N}(0,\Sigma)\) with \[\Sigma_{year} =
  \left[ {\begin{array}{cc}
    0.34 & 0.29 & -0.74 \\
    0.29 & 0.35 & -0.42\\
    -0.74 & -0.42 & 0.042\\
  \end{array} } \right]\]

\(K_{Pop} \sim \mathcal{N}(0,\Sigma)\) with \[\Sigma_{Pop} =
  \left[ {\begin{array}{cc}
    0.49 & -0.49 \\
    -0.49 & 0.0041 \\
  \end{array} } \right]\]

\subsubsection{Fécondité (nombre de
capitules)}\label{fuxe9condituxe9-nombre-de-capitules}

\[C(x)=\beta_0+\beta_1x\]

\[C(x)=2.95+2.07x\]

\(x\) : taille au temps \(t\)

\(\beta_0\) : intercept

\(\beta_1\): coefficients de la taille

\begin{Shaded}
\begin{Highlighting}[]
\CommentTok{\# Gyx0 \textless{}{-} function (y, x, a) \{}
\CommentTok{\#   fake \textless{}{-} fake\_data}
\CommentTok{\#   fake$age0 \textless{}{-} a \#Fixe l\textquotesingle{}age}
\CommentTok{\#   fake$Size0Mars \textless{}{-} unique(x) \#Fixe la taille}
\CommentTok{\#   sortie \textless{}{-} predict(Growthglm1, fake, allow.new.levels = T) \#Prédit les log(taille) à t+1 possibles }
\CommentTok{\#   sortie2 \textless{}{-} aggregate(sortie, list(fake$Size0Mars), mean) \#Calcule la moyenne des log(taille)t+1}
\CommentTok{\#   Sdev \textless{}{-} mean(residVar(Growthglm1))}
\CommentTok{\#   return(dnorm(y, mean = exp(sortie2$V1), sd = exp(Sdev))) \#Proba qu\textquotesingle{}un individu soit de taille y à t+1 sachant la moyenne et la variance}
\CommentTok{\# \}}
\CommentTok{\# }
\CommentTok{\# data\_predi\_growth \textless{}{-} fake\_data3 \%\textgreater{}\%}
\CommentTok{\#   filter(Pop=="Au",year==2000) \%\textgreater{}\% }
\CommentTok{\#   select({-}year,{-}Pop) \%\textgreater{}\% }
\CommentTok{\#   group\_by(Size0Mars,age0,Size1) \%\textgreater{}\% }
\CommentTok{\#   mutate(prob = Gyx0(x=Size0Mars, y=Size1,a=age0))}
\CommentTok{\# }
\CommentTok{\# data\_predi\_growth \%\textgreater{}\% }
\CommentTok{\#   ggplot(aes(x = Size0Mars, y = Size1)) +}
\CommentTok{\#   geom\_tile(aes(fill = prob)) +}
\CommentTok{\#   facet\_wrap(\textasciitilde{}age0)+}
\CommentTok{\#   scale\_fill\_viridis\_c(option = "inferno",direction={-}1) +}
\CommentTok{\#   theme\_classic()+}
\CommentTok{\#   labs(title = "Croissance prédite",}
\CommentTok{\#       x = "Size(t)",}
\CommentTok{\#       y = "Size(t+1)",}
\CommentTok{\#       fill = "Proba",}
\CommentTok{\#       color = "Proba")}
\end{Highlighting}
\end{Shaded}

\begin{Shaded}
\begin{Highlighting}[]
\NormalTok{resSurv1 }\OtherTok{\textless{}{-}} \FunctionTok{get\_residVar}\NormalTok{(Survglm11,}\AttributeTok{newdata =}\NormalTok{ fake\_data1)}

\NormalTok{fake\_data1 }\SpecialCharTok{\%\textgreater{}\%}
  \FunctionTok{mutate}\NormalTok{(}\AttributeTok{surv =}\NormalTok{ Survpredict11,}
         \AttributeTok{var =}\NormalTok{ resSurv1)}\SpecialCharTok{\%\textgreater{}\%}
  \FunctionTok{group\_by}\NormalTok{(Size0Mars,year) }\SpecialCharTok{\%\textgreater{}\%} 
  \FunctionTok{mutate}\NormalTok{(}\AttributeTok{surv\_predi =} \FunctionTok{mean}\NormalTok{(surv, }\AttributeTok{na.rm =} \ConstantTok{TRUE}\NormalTok{),}
         \AttributeTok{var =} \FunctionTok{mean}\NormalTok{(var)) }\SpecialCharTok{\%\textgreater{}\%}
  \FunctionTok{ggplot}\NormalTok{(}\FunctionTok{aes}\NormalTok{(}\AttributeTok{x =}\NormalTok{ Size0Mars, }\AttributeTok{y =}\NormalTok{ surv\_predi)) }\SpecialCharTok{+}
  \FunctionTok{geom\_ribbon}\NormalTok{(}\FunctionTok{aes}\NormalTok{(}\AttributeTok{ymin=}\NormalTok{surv\_predi}\SpecialCharTok{{-}}\NormalTok{var,}\AttributeTok{ymax=}\NormalTok{surv\_predi}\SpecialCharTok{+}\NormalTok{var))}\SpecialCharTok{+}
  \FunctionTok{geom\_line}\NormalTok{(}\FunctionTok{aes}\NormalTok{(}\AttributeTok{color=}\FunctionTok{as.factor}\NormalTok{(year))) }\SpecialCharTok{+}
  \FunctionTok{theme\_bw}\NormalTok{()}\SpecialCharTok{+}
  \FunctionTok{ylim}\NormalTok{(}\DecValTok{0}\NormalTok{, }\DecValTok{1}\NormalTok{) }\SpecialCharTok{+}
  \FunctionTok{xlim}\NormalTok{(}\DecValTok{0}\NormalTok{,}\DecValTok{9}\NormalTok{)}\SpecialCharTok{+}
  \FunctionTok{labs}\NormalTok{(}\AttributeTok{title =} \StringTok{"Survie prédites des plantules"}\NormalTok{,}
      \AttributeTok{subtitle =} \StringTok{"Moyenne de la survie sur les années et les populations"}\NormalTok{,}
      \AttributeTok{x =} \StringTok{"Size"}\NormalTok{,}
      \AttributeTok{y =} \StringTok{"Survival probability"}\NormalTok{)}
\end{Highlighting}
\end{Shaded}

\includegraphics{ModelSelectionReport_files/figure-latex/unnamed-chunk-17-1.pdf}

\subsection{Plots en fonction de la
taille}\label{plots-en-fonction-de-la-taille}

\includegraphics[width=0.5\linewidth]{ModelSelectionReport_files/figure-latex/unnamed-chunk-18-1}
\includegraphics[width=0.5\linewidth]{ModelSelectionReport_files/figure-latex/unnamed-chunk-18-2}
\includegraphics[width=0.5\linewidth]{ModelSelectionReport_files/figure-latex/unnamed-chunk-18-3}
\includegraphics[width=0.5\linewidth]{ModelSelectionReport_files/figure-latex/unnamed-chunk-18-4}
\includegraphics[width=0.5\linewidth]{ModelSelectionReport_files/figure-latex/unnamed-chunk-18-5}
\includegraphics[width=0.5\linewidth]{ModelSelectionReport_files/figure-latex/unnamed-chunk-18-6}
\includegraphics[width=0.5\linewidth]{ModelSelectionReport_files/figure-latex/unnamed-chunk-18-7}
\includegraphics[width=0.5\linewidth]{ModelSelectionReport_files/figure-latex/unnamed-chunk-18-8}
\includegraphics[width=0.5\linewidth]{ModelSelectionReport_files/figure-latex/unnamed-chunk-18-9}

On voit clairement que la probabilité de survie augmente quand la plante
grandit, jusqu'à atteindre un plateau pour les rosettes. La survie d'une
plante de petite taille est assez basse, peu importe qu'elle soit d'age
1 ou plus. On ne peut pas vraiment conclure sur la survie des plantules
de grande taille (\textgreater5) du fait de la très faible quantité de
données pour ces classes de tailles (27 observations).

On voit que la probabilité de floraison augmente fortement avec la
taille. De même que pour la survie, on ne peut rien dire de la floraison
des plantules de grandes tailles.

\subsubsection{Plots en fonction de
l'age}\label{plots-en-fonction-de-lage}

\includegraphics[width=0.5\linewidth]{ModelSelectionReport_files/figure-latex/unnamed-chunk-19-1}
\includegraphics[width=0.5\linewidth]{ModelSelectionReport_files/figure-latex/unnamed-chunk-19-2}
\includegraphics[width=0.5\linewidth]{ModelSelectionReport_files/figure-latex/unnamed-chunk-19-3}
\includegraphics[width=0.5\linewidth]{ModelSelectionReport_files/figure-latex/unnamed-chunk-19-4}

\subsubsection{Autres plots}\label{autres-plots}

\includegraphics[width=0.5\linewidth]{ModelSelectionReport_files/figure-latex/unnamed-chunk-20-1}
\includegraphics[width=0.5\linewidth]{ModelSelectionReport_files/figure-latex/unnamed-chunk-20-2}

\section{Modèles par BIC}\label{moduxe8les-par-bic}

\begin{Shaded}
\begin{Highlighting}[]
\NormalTok{Survglm11 }\OtherTok{\textless{}{-}} \FunctionTok{fitme}\NormalTok{(SurvieMars }\SpecialCharTok{\textasciitilde{}} \DecValTok{1}\SpecialCharTok{+} \FunctionTok{poly}\NormalTok{(Size0Mars,}\DecValTok{3}\NormalTok{) }\SpecialCharTok{+}\NormalTok{ (}\DecValTok{1}\SpecialCharTok{|}\NormalTok{year) }\SpecialCharTok{+}\NormalTok{ (}\DecValTok{1}\SpecialCharTok{|}\NormalTok{Pop),}
                  \AttributeTok{family=}\NormalTok{binomial,}
                  \AttributeTok{data=}\NormalTok{survdata1,}
                  \AttributeTok{method=}\StringTok{"PQL/L"}\NormalTok{)}
\FunctionTok{extractBIC}\NormalTok{(Survglm11,}\AttributeTok{ntot=}\FunctionTok{length}\NormalTok{(survdata1),}\AttributeTok{N=}\DecValTok{28}\SpecialCharTok{*}\DecValTok{6}\NormalTok{)}
\end{Highlighting}
\end{Shaded}

\begin{verbatim}
## [1] 4390.364
\end{verbatim}

\begin{Shaded}
\begin{Highlighting}[]
\NormalTok{Survglm12 }\OtherTok{\textless{}{-}} \FunctionTok{fitme}\NormalTok{(SurvieMars }\SpecialCharTok{\textasciitilde{}} \DecValTok{1} \SpecialCharTok{+} \FunctionTok{bs}\NormalTok{(Size0Mars,}\AttributeTok{df=}\DecValTok{3}\NormalTok{,}\AttributeTok{degree=}\DecValTok{2}\NormalTok{) }\SpecialCharTok{+}\NormalTok{ (age0}\SpecialCharTok{|}\NormalTok{year) }\SpecialCharTok{+}\NormalTok{ (}\DecValTok{1}\SpecialCharTok{|}\NormalTok{Pop),}
                  \AttributeTok{family=}\NormalTok{binomial,}
                  \AttributeTok{data=}\NormalTok{survdata2,}
                  \AttributeTok{method=}\StringTok{"PQL/L"}\NormalTok{)}
\FunctionTok{extractBIC}\NormalTok{(Survglm12,}\AttributeTok{ntot=}\FunctionTok{length}\NormalTok{(survdata2),}\AttributeTok{N=}\DecValTok{28}\SpecialCharTok{*}\DecValTok{6}\NormalTok{)}
\end{Highlighting}
\end{Shaded}

\begin{verbatim}
## [1] 2340.671
\end{verbatim}

\begin{Shaded}
\begin{Highlighting}[]
\NormalTok{Flowglm1 }\OtherTok{\textless{}{-}} \FunctionTok{fitme}\NormalTok{(Flowering0 }\SpecialCharTok{\textasciitilde{}}  \DecValTok{1} \SpecialCharTok{+} \FunctionTok{poly}\NormalTok{(Size0Mars,}\DecValTok{3}\NormalTok{) }\SpecialCharTok{+} \FunctionTok{poly}\NormalTok{(age0,}\DecValTok{2}\NormalTok{) }\SpecialCharTok{+}\NormalTok{ (age0}\SpecialCharTok{|}\NormalTok{Pop),}
                 \AttributeTok{family=}\NormalTok{binomial,}
                 \AttributeTok{data=}\NormalTok{centauree\_data, }\AttributeTok{method=}\StringTok{"PQL/L"}\NormalTok{)}
\FunctionTok{extractBIC}\NormalTok{(Flowglm1,}\AttributeTok{ntot=}\FunctionTok{length}\NormalTok{(centauree\_data),}\AttributeTok{N=}\DecValTok{6}\NormalTok{)}
\end{Highlighting}
\end{Shaded}

\begin{verbatim}
## [1] 853.1186
\end{verbatim}

\begin{Shaded}
\begin{Highlighting}[]
\NormalTok{Growthglm1 }\OtherTok{\textless{}{-}} \FunctionTok{fitme}\NormalTok{(}\FunctionTok{log}\NormalTok{(Size1Mars) }\SpecialCharTok{\textasciitilde{}} \DecValTok{1} \SpecialCharTok{+} \FunctionTok{poly}\NormalTok{(}\FunctionTok{log}\NormalTok{(Size0Mars),}\DecValTok{3}\NormalTok{) }\SpecialCharTok{+} \FunctionTok{poly}\NormalTok{(age0,}\DecValTok{3}\NormalTok{) }\SpecialCharTok{+}\NormalTok{ (}\FunctionTok{log}\NormalTok{(Size0Mars)}\SpecialCharTok{|}\NormalTok{year) }\SpecialCharTok{+}\NormalTok{ (}\FunctionTok{log}\NormalTok{(Size0Mars)}\SpecialCharTok{|}\NormalTok{Pop),}
                    \AttributeTok{data=}\NormalTok{growthdata)}
\FunctionTok{extractBIC}\NormalTok{(Growthglm1,}\AttributeTok{ntot=}\FunctionTok{length}\NormalTok{(growthdata),}\AttributeTok{N=}\DecValTok{6}\SpecialCharTok{*}\DecValTok{28}\NormalTok{)}
\end{Highlighting}
\end{Shaded}

\begin{verbatim}
## [1] 4113.802
\end{verbatim}

\begin{Shaded}
\begin{Highlighting}[]
\NormalTok{Pltglm1 }\OtherTok{\textless{}{-}} \FunctionTok{fitme}\NormalTok{(Size0Mars }\SpecialCharTok{\textasciitilde{}} \DecValTok{1} \SpecialCharTok{+}\NormalTok{ (}\DecValTok{1}\SpecialCharTok{|}\NormalTok{year) }\SpecialCharTok{+}\NormalTok{ (}\DecValTok{1}\SpecialCharTok{|}\NormalTok{Pop) }\SpecialCharTok{+}\NormalTok{ (}\DecValTok{1}\SpecialCharTok{|}\NormalTok{Pop}\SpecialCharTok{:}\NormalTok{year),}
                 \AttributeTok{data=}\NormalTok{plantule\_data,}
                 \AttributeTok{family =} \FunctionTok{Gamma}\NormalTok{(log))}
\FunctionTok{extractBIC}\NormalTok{(Pltglm1, }\AttributeTok{ntot =} \FunctionTok{length}\NormalTok{(plantule\_data), }\AttributeTok{N=}\DecValTok{28}\SpecialCharTok{*}\DecValTok{6}\SpecialCharTok{*}\DecValTok{147}\NormalTok{)}
\end{Highlighting}
\end{Shaded}

\begin{verbatim}
## [1] 8473.623
\end{verbatim}

\begin{Shaded}
\begin{Highlighting}[]
\NormalTok{Cptlglm1 }\OtherTok{\textless{}{-}} \FunctionTok{fitme}\NormalTok{(Cptl0 }\SpecialCharTok{\textasciitilde{}} \DecValTok{1} \SpecialCharTok{+}\NormalTok{ Size0Mars,}
                  \AttributeTok{data =}\NormalTok{ cptldata)}
\FunctionTok{extractBIC}\NormalTok{(Cptlglm1,}\AttributeTok{ntot =} \FunctionTok{length}\NormalTok{(cptldata),}\AttributeTok{N=}\DecValTok{1}\NormalTok{)}
\end{Highlighting}
\end{Shaded}

\begin{verbatim}
## [1] 826.5143
\end{verbatim}

\subsection{Plots en fonction de la
taille}\label{plots-en-fonction-de-la-taille-1}

\includegraphics[width=0.5\linewidth]{ModelSelectionReport_files/figure-latex/unnamed-chunk-23-1}
\includegraphics[width=0.5\linewidth]{ModelSelectionReport_files/figure-latex/unnamed-chunk-23-2}
\includegraphics[width=0.5\linewidth]{ModelSelectionReport_files/figure-latex/unnamed-chunk-23-3}
\includegraphics[width=0.5\linewidth]{ModelSelectionReport_files/figure-latex/unnamed-chunk-23-4}
\includegraphics[width=0.5\linewidth]{ModelSelectionReport_files/figure-latex/unnamed-chunk-23-5}

Les modèle choisis par BIC sont globalement similaires que pour ceux
choisis par AIC. On constate quand même des différences pour la survie
des rosettes et pour la croissance. Pour la survie des rosettes, le
modèle choisi propose une diminution de la survie pour les plantes de
grande taille. Cette incertitude avec le modèle AIC correspond avec
l'incertitude qu'on a pour ces classes de taille qui comportent peu de
données. Pour la croissance, l'effet de réduction de la croissance pour
les plantes de grande taille se retrouve dans les deux modèles (AIC et
BIC), cependant cet effet est moins fort pour le modèle BIC.

\subsubsection{Plots en fonction de
l'age}\label{plots-en-fonction-de-lage-1}

\includegraphics[width=0.5\linewidth]{ModelSelectionReport_files/figure-latex/unnamed-chunk-24-1}
\includegraphics[width=0.5\linewidth]{ModelSelectionReport_files/figure-latex/unnamed-chunk-24-2}
\includegraphics[width=0.5\linewidth]{ModelSelectionReport_files/figure-latex/unnamed-chunk-24-3}
\includegraphics[width=0.5\linewidth]{ModelSelectionReport_files/figure-latex/unnamed-chunk-24-4}

La sélection par le BIC réduit le nombre de variables explicatives. On
peut le voir notamment pour la survie des rosettes en fonction de l'age.
Le modèle BIC ne choisit pas une fonction de survie qui dépend de l'age,
il y a seulement un effet aléatoire de l'age sur l'année.

\subsubsection{Autres plots}\label{autres-plots-1}

\includegraphics[width=0.5\linewidth]{ModelSelectionReport_files/figure-latex/unnamed-chunk-25-1}

\end{document}
